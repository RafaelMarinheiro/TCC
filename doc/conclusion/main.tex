%!TEX root = ../main.tex
\chapter{Conclusão}

Nesse trabalho nós estudamos o problema de Síntese de Áudio Realístico para aplicações virtuais. Exploramos o estado-da-arte das técnicas de síntese, considerando a diferença entre os Métodos por Amostragem e os Métodos por Simulação Física.

Nós deselvovemos um método eficiente para aproximar a solução da Equação de Helmholtz em GPU. Utilizamos esse método para acelerar uma parte do pipeline de síntese. Comparamos a nossa implementação com implementação anteriores obtendo resultados bastante satisfatórios: O nosso método foi mais de $30\times$ mais rápido que os métodos anteriores. Além disso, fizemos um teste A/B com voluntários que mostrou que a qualidade do áudio gerado foi melhor.

Acreditamos que essa contribuição representa um passo importante para a área. Os métodos de síntese por modelagem física ainda são limitados pelo seu custo computacional. O uso de algoritmos mais eficientes que explorem a paralelização em GPU pode ser a chave para popularizar aplicações que utilizem esses métodos. Acreditamos que isso contribuirá para a criação de aplicações audiovisuais que propriciem experiências mais imersivas e satisfatórias para os usuários.

Mais importante do que os resultados apresentados é a própria fundamentação dos conceitos envolvidos. O conhecimento de equações diferenciais, de métodos eficientes de solução e de paralelização em GPU pode ser utilizado nas mais diversas áreas.

\section{Trabalhos futuros}

Ainda existem muitas limitações com os métodos do estado-da-arte. O custo computacional da fase de pré-processamento e também da fase de tempo de execução são consideravelmente altos. Isso inviabiliza a simulação de objetos maiores e também dificulta a simulação em tempo real em computadores com configurações modestas. Trabalhos futuros podem tentar explorar técnicas de discretização adaptativas para lidar com domínios maiores e também pode tentar explorar técnicas de GPU para paralelizar as demais etapas do pipeline.