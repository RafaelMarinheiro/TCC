\section{Som e Acústica}

Acústica é a ciência que estuda a som. É ciência que estuda a sua origem e propagação, seja ela em espaços abertos ou espaços fechados \cite{kuttruff2007acoustics}.

\subsection{Equação da Onda}

No estudo de acústica, a principal equação é a \emph{Equação da Onda}. Considerando um meio dispersante, a pressão do meio é governada pela seguinte equação diferencial:

\begin{eqnarray}
\frac{\partial^2 \rho(u, t)}{\partial t^2} &=& dc^2 \nabla^2 \rho(u, t)\label{acousticequation} \\
&\text{onde}&\nonumber\\
t \in \mathbb{R} &\Rightarrow& \text{Tempo em segundos }(s) \nonumber\\
u \in \mathbb{R}^3 &\Rightarrow& \text{Posição em metros }(m) \nonumber\\
c \in \mathbb{R} &\Rightarrow& \text{Velocidade de Propagação }(\nicefrac{m}{s}) \nonumber\\
c \in \mathbb{R} &\Rightarrow& \text{Densidade } \nicefrac{kg}{m^3} \nonumber\\
\rho(u, t) \in \mathbb{R} &\Rightarrow& \text{Pressão acústica }(Pa) \nonumber
\end{eqnarray}

Na equação \eqref{acousticequation}, o operador $\nabla^2$ corresponde ao operador Laplaciano Espacial. Isto é:

\begin{equation}
	\nabla^2 \rho(u, t) = \Delta \rho(u, t) = \frac{\partial^2 \rho(u, t)}{\partial x^2} + \frac{\partial^2 \rho(u, t)}{\partial y^2} + \frac{\partial^2 \rho(u, t)}{\partial z^2}
\end{equation}

A equação \eqref{acousticequation} é uma equação diferencial parcial linear hiperbólica de segunda ordem. Ela é chamada de equação governante pois, determinadas as condições inicias e as condições de contorno, os demais resultados são obtidos como consequência dela.

\begin{figure}[ht]
	\centering
	\input{mathematicalbackground/wave_field.tikz}
	\caption[Pressão acústica gerada uma fonte pontual]{Pressão acústica gerada por uma fonte pontual no centro do meio. A fonte e o ambiente determinam as condições de contorno e a pressão acústica nos demais pontos é governada pela Equação da Onda \eqref{acousticequation}}\label{wavefield}
\end{figure}

\subsubsection{Fontes Sonoras e Ambientes Acústicos}

Em acústica, existem duas entidades essenciais: a \emph{fonte sonora}, responsável por gerar o som, e o \emph{ambiente}, responsável por transmití-lo. Matemáticamente essas entidades definem exatamente as condições iniciais e as condições de contorno da Equação da Onda.

Uma equação diferencial é definida dentro um domínio, usualmente denotado por $\Omega$. O domínio pode ser um ambiente fechado (sala de estar ou um auditório) ou um espaço aberto (campo ou um deserto). As condições de contorno são equações que delimitam o comportamento da pressão acústica dentro do domínio $\Omega$. Os dois tipos mais comuns de condições de contorno são as condições de contorno de Dirichlet e as condições de contorno de Neumann.

As condições de contorno de Dirichlet são condições de contorno da forma:

\begin{equation}
	\rho(u, t) = f(u, t)
\end{equation}

Essas condições de contorno forçam que o valor de determinados pontos obedeça alguma função já conhecida. Em acústica, as condições de contorno de Dirichlet normalmente são utilizadas para definir fontes sonoras. Se uma determinada fonte sonora localizada no ponto $u_{f}$ reproduz um som $s_f(t)$, podemos adicionar a condição de contorno $\rho(u_{f}, t) = s_f(t)$.

As condições de contorno de Neumann são condições de contorno da forma:

\begin{eqnarray}
\frac{\partial\rho(u, t)}{\partial n} &=& f(u, t)\\
&\text{onde}& \nonumber \\
n \in \mathbb{R}^3 &\Rightarrow& \text{Vetor normal}\nonumber 
\end{eqnarray}

Essas condições de contorno forçam que o valor da derivada da pressão em determinados pontos obedeça alguma função já conhecida. As condições de contorno de Dirichlet são utilizadas, por exemplo, para determinar o comportamento na interface entre diferentes meios. Se um ponto $u_{b}$ encontra-se na borda de uma superfície rígida fixa de normal $n_b$, podemos adicionar a condição de contorno $\nicefrac{\partial\rho(u_b, t)}{\partial n_b} = 0$.

\subsection{Radiação acústica e a Equação de Helmholtz}

Dadas as condições iniciais e as condições de contorno, a equação da onda nos dá o valor exato da pressão acústica em função da posição e do tempo. A dependência com o tempo, no entanto, nos força a resolver a equação da onda durante a simulação. Esse processo tem um alto custo computacional, portanto é preferível encontrar uma outra solução para esse problema.


O caso mais comum de síntese acústica consiste no problema das fontes sonoras: dada uma fonte sonora na posição $u_f$ emitindo um som $f(t)$, qual a pressão acústica $\rho(u, t)$ num determinado ponto $u$ no tempo $t$?

Podemos tentar assumir que a pressão acústica $\rho(u, t)$ depende de duas funções independentes. Isto, é, podemos tentar assumir que:

\begin{equation}
	\rho(u, t) = T(u)F(t)
\end{equation}

onde $T(u)$ seria a amplitude no ponto $u$ e $F(t)$ seria uma função que depende apenas som emitido pelas fontes sonoras. Com essa separação, se a fonte passasse a emitir o som $f^*(t)$, precisaríamos apenas computar a função $F^*(t)$ para obter a solução $\rho^*(u, t) = T(u)F^*(t)$. 

A função $T(u)$ também é chamada de \emph{Radiação Acústica} ou \emph{Transferência Acústica}, pois ela mede quanta energia acústica foi transferida a partir das fontes sonoras. A equação governante para a Radiação Acústica é a \emph{Equação de Helmholtz}:

\begin{eqnarray}
	\nabla^2 T(u) + k^2T(u) &=& 0\label{helmholtzequation}\\
&\text{onde}&\nonumber\\
k \in \mathbb{R} &\Rightarrow& \frac{\omega}{c}\rightarrow\text{ Número de Onda (1/m)} \nonumber\\
u \in \mathbb{R}^3 &\Rightarrow& \text{ Posição em metros(m)} \nonumber\\
T(u) \in \mathbb{C} &\Rightarrow& \text{ Amplitude em metros(m)} \nonumber
\end{eqnarray}

\begin{figure}[ht]
	\centering
	\input{mathematicalbackground/helmholtz_abs.tikz}
	\caption[Radiação acústica gerada uma fonte pontual]{Radiação acústica gerada por uma fonte pontual no centro do meio. A fonte e o ambiente determinam as condições do contorno e a radiação acústica nos demais pontos é governada pela Equação de Helmholtz}\label{helmholtz}
\end{figure}

Dois fatos são notáveis na Equação de Helmholtz: A constante $k = \frac{\omega}{c}$ depende da frequência $\omega$ da fonte sonora. Por tal razão, a Equação de Helmholtz é comumente chamada de Equação da Onda no domínio da frequencia.

Note também que a função $T(u)$ tem como contra-domínio o conjunto dos números complexos $\mathbb{C}$. O uso de números complexos nesse caso é útil para representarmos a fase da onda sonora. Se escrevermos $T(u) = A(u)e^{i\phi(u)}$ e $F(t) = A_0e^{i\omega t}$, obteremos:

\begin{eqnarray}
	\rho(u, t) =& Re\left( T(u)F(t) \right) \nonumber\\
	=& Re\left(A_0A(u) e^{i(\omega t + \phi(u)}\right) \nonumber\\
	=& Re\left(A_0A(u)(\cos (\omega t + \phi(u)) + i\sin (\omega t + \phi(u))\right) \nonumber\\
	=& A_0A(u)\cos \theta(\omega t + \phi(u))
\end{eqnarray}

Nesse caso, dizemos que a função $\phi(u)$ é a fase da onda sonora.

\subsubsection{Aproximação em Far-Field}

Considere um sistema com algumas fontes sonoras. Seja $B(r)$ a esfera de raio $r$ cujo centro está na origem. Seja $r_{near}$ tal que $B(r_{near})$ contenha todas as fontes sonoras do sistema. Dizemos que a região $B(r_{near})$ é o Near-Field (Campo Próximo) do sistema e que a região externa é o Far-Field (Campo Longíquo) do sistema (Ver \figref{fig:farfield}).

\begin{figure}[ht]
	\centering
	\input{mathematicalbackground/farfield.tikz}
	\caption[Radiação de Near-Field e Far-Field]{Radiação de Near-Field e Far-Field.}\label{fig:farfield}
\end{figure}

Essa separação entre o Near-Field e o Far-Field é importante para efeitos práticos. Toda a radiação acústica é gerada dentro do Near-Field e depois é dissipada para o Far-Field que, por sua vez, não deve gerar radiação acústica. Isso é matematicamente representado pela Condição de Contorno de Sommerfeld, também conhecida por \emph{Condição de Radiação de Sommerfeld}:

\begin{equation}
	\lim_{|u| \rightarrow \infty} |u| \left(\frac{\partial}{\partial |u|} - ik \right) T(u) = 0 \label{eq:sommerfeld_condition}
\end{equation}

Além disso, também observa-se uma diferença entre a complexidade do Near-Field quando comparado ao Far-Field. Dentro do Near-Field a interação entre as fontes sonoras e os demais objetos faz com que a radiação acústica seja extremamente complexo. Entretanto, a radiação acústica no Far-Field tem uma estrutura muito mais simples.

Seja $\Gamma$ a superfície do Near-Field e seja $u$ um ponto no Far-Field. A radiação $T(u)$ pode ser calculada pela \emph{Integral de Kirchhoff}:

\begin{equation}
	T(u) = \int_{\Gamma} \left[G(u, v)\frac{\partial T}{\partial n}(v) - \frac{\partial G}{\partial n}(u, v)T(v) \right] d\Gamma_v
	\label{kirchhorff_integral}
\end{equation}

onde $G(u, v)$ é a Função de Green da Equação Helmholtz:

\begin{equation}
	G(u, v) = \frac{e^{ik\norm{u-v}}}{4\pi \norm{u-v}}
\end{equation}

Utilizando a fórmula \eqref{kirchhorff_integral}, vemos que é necessário apenas calcular a Radiação Acústica no Near-Field para definirmos a solução para o Far-Field.