%!TEX root = ../main.tex
\chapter{Síntese de Áudio Realístico}

O problema de Síntese de Aúdio Realístico pode ser definido da seguinte maneira: Dada a descrição de uma cena, como podemos sintetizar o áudio correspondente? Para entedermos melhor esse definição, precisamos entender o que seria a descrição de uma cena e também o que se entende por aúdio correspondente.

A descrição da cena é o conjunto de informações necessárias para descrever a composição e o estado dos objetos presentes. A descrição da cena pode ser dada de modo explícito ou implícito. Uma descrição explícita possui todos os parâmetros dos objetos. Um exemplo seria: ``No tempo $t$, a cena contém um prato de cerâmica está na posição $p_a$ com velocidade $v_a$ e uma chate de aço na posição $p_b$ com velocidade $v_b$''. Uma descrição implícita é aquela que não apresenta os parâmetros explicitamente. Um exemplo de descrição implícita seria um vídeo com os objetos. Nesse caso, o vídeo deveria ser processado para que a descrição explícita fosse extraída.

Dizemos que um áudio satisfaz a descrição da cena (ou que ele corresponde à descrição da cena) se ele é suficientemente parecido com o áudio que seria produzido se a cena fosse simulada no mundo real. O grau de fidelidade varia de acordo com a aplicação. Em aplicações de Engenharia, por exemplo, pode-se exigir um alto grau de fidelidade. Em aplicações lúdicas, o grau de fidelidade pode ser menor. No nosso caso, consideramos que o áudio é fiel o suficiente se um humano o julga plausível. Estudos mostram que humanos são particulamente sensíveis à mudanças drásticas no timbre (e.g.: som metálico gerado por um objeto de madeira e vice-versa) e intensidade (e.g.: som muito alto para uma ação fraca e vice-versa) e à falhas de sincronização\cite{bonebright2012were}.

Uma boa solução para problema deve ser capaz de gerar áudio que satisfaça a descrição da cena. A solução também deve ser de baixo custo e também deve exigir pouco tempo.

O trabalhos da área podem ser divididos em Métodos por Modelagem Física e Métodos por Amostragem.

\section{Métodos Orientados à Amostras}
Métodos orientados à amostras são métodos que utilizam amostras de sons como entrada para o algoritmo de síntese. Um procedimento comum nessa classe de métodos é a extração de características das amostras originais e a extrapolação dessas características na etapa de síntese.

O trabalho de \cite{lloyd2011sound}, por exemplo, faz uma análise nas amostras sonoras para identificar as frequências dominantes. As frequências dominantes assemelham-se às frequências naturais do objeto. Durante o tempo de execução, as frequências dominantes são somadas poderando cada uma com pesos gerados aleatoriamente. Os sons gerados não são fisicamente plausíveis e também não podem ser estendidos de maneira adequada para objetos distintos. Porém ela é uma abordagem computacionalmente barata que oferece um certo grau de dinamismo na acústica do ambiente virtual. Por tais razões, ela é ideal para ser incorporada à jogos em tempo real.

Os trabalhos de \cite{ren2013example} e de \cite{sterling2016interactive} utilizam uma abordagem mista. Os dois utilizam as amostras sonoras para encontrar os parâmetros do material. Esses parâmetros são então utilizados como entrada em métodos baseados em física. Para aprender esses parâmetros, o som e o modelo virtual de um objeto real são utilizados como entrada do algoritmo. Os resultados gerados são plausíveis embora fisicamente incorretos. 

Redes neurais também são comuns na área de síntese de aúdio. Embora o seu uso seja mais comum na área de síntese de fala \cite{rahim1993use, ling2015deep}, trabalhos recentes tem explorado técnicas para gerar sons de objetos. O trabalho de \cite{visualIndicatedSounds} utiliza Deep Learning para extrair características do áudio de amostras de vídeos. Com essas características, os autores conseguem sintetizar áudio plausível para outras cenas sem áudio.


\section{Métodos Baseados em Física}

Os métodos baseados em física são aqueles que tentam, através das propriedades físicas dos materiais, sintetizer o som adequado. Em outras áreas, como Engenharia Mecânica ou Engenharia Aeronáutica, o estudo da radiação acústica gerada por materiais já é uma prática comum \cite{pierce1981acoustics}. No entanto, os métodos utilizados nessas áreas costumam considerar apenas cenas estáticas, o que inviabiliza a interação com o usuário.

O trabalho de \cite{van2001foleyautomatic} foi o primeiro a considerar o uso de métodos physically-based em aplicações interativas. Os autores desenvolveram um processo simples para aplicações interativas: Em tempo de pré-processamento, os modos de vibração do objetos eram calculados e armazenados. Durante a execução, os objetos eram simulados e o estado $q_i(t)$ de cada modo de vibração era considerado utilizando a equação \eqref{linear_modal_eqn}. O som final gerado pelo objeto era simplesmente a soma dos modos de vibração:

\begin{equation}
	s(t) = \sum_i q_i(t)
\end{equation}

Essa abordagem conseguia gerar som plausível em tempo real, mas muitas nuances do som eram ignoradas. A distância e o ângulo do objeto em relação ao ouvinte, por exemplo, têm uma grande influência no som final. O trabalho de \cite{james2006precomputed} então introduziu o uso de Radiação Acústica no processo de síntese (Ver \figref{fig:pipeline_overview}). Nesse trabalho, a função de Radiação Acústica $R_i(u)$ era calculada para cada modo de vibração do objeto. Em tempo de execução, a posição do objeto em relação ao ouvinte era considerada. Com essa abordagem, o som final consideraria os modos de vibração ponderados pela radiação acústica: 

\begin{equation}
	s(t) = \sum_i |R_i(u_{ouvinte}-u_{objeto})| q_i(t)
\end{equation}

\begin{figure}[ht]
\begin{subfigure}{\textwidth}
	\centering
	\input{relatedwork/pipeline_offline.tikz}
	\caption{Fase de Pré-Processamento}\label{pipeline_offline}
\end{subfigure}
\begin{subfigure}{\textwidth}
	\centering
	\input{relatedwork/pipeline_online.tikz}
	\caption{Pipeline em Tempo de Execução}\label{pipeline_online}
\end{subfigure}
\caption[Overview do pipeline]{\figref{pipeline_offline} plots of....}
\label{fig:pipeline_overview}
\end{figure}

O pipeline desenvolvido por \cite{james2006precomputed} é utilizado como base para os trabalhos mais recentes na área.\cite{zheng2010rigid}, por exemplo, extenderam o modelo para também sintetizar o som de fraturas.

\cite{zheng2011toward} melhoraram o modelo de contato utilizado na simulação do objeto. A qualidade do som final depende muito das forças de contato $f_{ext}$. Para capturar melhor as nuances do contato, eles passaram a utilizar um simulador numericamente estável \cite{kaufman2008staggered} e também aumentaram a resolução da simulação para a resolução acústica (44KHz). Como resultado, a qualidade do áudio foi drasticamente melhorada mas o tempo necessário para realizar a síntese também aumentou.

Para realizar a síntese, os modos de vibração tem que ser utilizados em tempo de execução. Para objetos maiores, a quantidade de memória necessária para armazená-los pode inviabilizar a simulação. \cite{langlois2014eigenmode} desenvolveram um método para comprimir os modos de vibração. Isso possibilitou utilizar objetos maiores e mais detalhados sem impacto no tempo de síntese. 



