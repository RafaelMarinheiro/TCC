\section{Métodos Physically-Based}

Os Métodos Physically-Based, ou métodos baseados em física, são aqueles que tentam, através das propriedades físicas dos materiais, sintetizer o som adequado. Em outras áreas, como Engenharia Mecânica ou Engenharia Aeronáutica, o estudo da radiação acústica gerada por materiais já é uma prática comum \cite{pierce1981acoustics}. No entanto, os métodos utilizados nessas áreas costumam considerar apenas cenas estáticas, o que inviabiliza a interação com o usuário.

O trabalho de \cite{van2001foleyautomatic} foi o primeiro a considerar o uso de métodos physically-based em aplicações interativas. Os autores desenvolveram um processo simples para aplicações interativas: Em tempo de pré-processamento, os modos de vibração do objetos eram calculados e armazenados. Durante a execução, os objetos eram simulados e o estado $q_i(t)$ de cada modo de vibração era considerado utilizando a equação \eqref{linear_modal_eqn}. O som final gerado pelo objeto era simplesmente a soma dos modos de vibração:

\begin{equation}
	s(t) = \sum_i q_i(t)
\end{equation}

Essa abordagem conseguia gerar som plausível em tempo real, mas muitas nuances do som eram ignoradas. A distância e o ângulo do objeto em relação ao ouvinte, por exemplo, têm uma grande influência no som final. O trabalho de \cite{james2006precomputed} então introduziu o uso de Radiação Acústica no processo de síntese (Ver \figref{fig:pipeline_overview}). Nesse trabalho, a função de Radiação Acústica $T_i(u)$ era calculada para cada modo de vibração do objeto. Em tempo de execução, a posição do objeto em relação ao ouvinte era considerada. Com essa abordagem, o som final consideraria os modos de vibração ponderados pela radiação acústica: 

\begin{equation}
	s(t) = \sum_i |T_i(u_{ouvinte}-u_{objeto})| q_i(t)
\end{equation}

\begin{figure}[ht]
\begin{subfigure}{\textwidth}
	\centering
	\input{relatedwork/pipeline_offline.tikz}
	\caption{Fase de Pré-Processamento}\label{pipeline_offline}
\end{subfigure}
\begin{subfigure}{\textwidth}
	\centering
	\input{relatedwork/pipeline_online.tikz}
	\caption{Fase em Tempo de Execução}\label{pipeline_online}
\end{subfigure}
\caption[Pipeline de Síntese de Aúdio Realístico]{Pipeline de Síntese de Aúdio Realístico.}
\label{fig:pipeline_overview}
\end{figure}

O pipeline desenvolvido em \cite{james2006precomputed} é utilizado como base para os trabalhos mais recentes na área.\cite{zheng2010rigid}, por exemplo, estendeu o modelo para também sintetizar o som de fraturas.

\cite{zheng2011toward} melhorou o modelo de contato utilizado na simulação do objeto. A qualidade do som final depende muito das forças de contato $f_{ext}$. Para capturar melhor as nuances do contato, eles passaram a utilizar um simulador numericamente estável \cite{kaufman2008staggered} e também aumentaram a resolução da simulação para a resolução acústica (44KHz). Como resultado, a qualidade do áudio foi drasticamente melhorada mas o tempo necessário para realizar a síntese também aumentou.

Para realizar a síntese, os modos de vibração tem que ser utilizados em tempo de execução. Para objetos maiores, a quantidade de memória necessária para armazená-los pode inviabilizar a simulação. Em \cite{langlois2014eigenmode} um método foi desenvolvido para comprimir os modos de vibração. Isso possibilitou utilizar objetos maiores e mais detalhados sem impacto no tempo de síntese. 