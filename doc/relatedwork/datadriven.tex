\section{Métodos por Amostragem}
Métodos Data-Driven ou Métodos por Amostragem são métodos que utilizam amostras sonoras como entrada para o algoritmo de síntese. As amostras sonoras normalmente são obtidas por artistas de Foley. Essas amostras comumente passar por uma etapa de pós-processamento antes de serem utilizadas. Essa etapa de pós-processamento normalmente consiste de métodos de extração e extrapolação de características para tentar adaptar o som da amostra para a cena.

O trabalho de \cite{lloyd2011sound}, por exemplo, faz uma análise nas amostras sonoras para identificar as frequências dominantes. As frequências dominantes assemelham-se às frequências naturais do objeto. Durante o tempo de execução, as frequências dominantes são somadas poderando cada uma com pesos gerados aleatoriamente. Os sons gerados não são fisicamente plausíveis e também não podem ser estendidos de maneira adequada para objetos distintos. Porém ela é uma abordagem computacionalmente barata que oferece um certo grau de dinamismo na acústica do ambiente virtual. Por tais razões, ela é ideal para ser incorporada à jogos em tempo real.

Os trabalhos de \cite{ren2013example} e de \cite{sterling2016interactive} utilizam uma abordagem mista. Os dois utilizam as amostras sonoras para encontrar os parâmetros do material. Esses parâmetros são então utilizados como entrada em métodos baseados em física. Para aprender esses parâmetros, o som e o modelo virtual de um objeto real são utilizados como entrada do algoritmo. Os resultados gerados são plausíveis embora fisicamente incorretos. 

Redes neurais também são comuns na área de síntese de aúdio. Embora o seu uso seja mais comum na área de síntese de fala \cite{rahim1993use, ling2015deep}, trabalhos recentes tem explorado técnicas para gerar sons de objetos. O trabalho de \cite{visualIndicatedSounds} utiliza Deep Learning para extrair características do áudio de amostras de vídeos. Com essas características, os autores conseguem sintetizar áudio plausível para outras cenas sem áudio.

Os métodos por amostragem tem diversas vantagens. Como amostras reais de aúdio são utilizadas, o som gerado normalmente tem boa qualidade. Esse tipo de método costuma ser computacionalmente barato, o que torna fácil a sua integração em diversas aplicações. As suas desvantagens também são aparentes: embora o áudio seja de boa qualidade, ele pode ser incompatível com a cena apresentada. Em aplicações dinâmicas, esse tipo de método tem problema para adaptar o som, o que pode criar experiências acústicas repetitivas.
